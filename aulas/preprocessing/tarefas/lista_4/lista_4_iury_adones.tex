%{{{ Preamble
\documentclass[12pt,a4paper]{article}
%}}}
%{{{ Packages
%{{{ geometry
\usepackage[
    bottom=3cm,
    left=2cm,
    right=2cm,
    top=3cm,
]{geometry}
%}}}
%{{{ bibliography
\usepackage[
    backend=biber,
    defernumbers=true,
    encoding=latin1,
    style=abnt,
]{biblatex}
\usepackage[autostyle]{csquotes}
%}}}
%{{{ type input font
\usepackage[T1]{fontenc}
\usepackage[brazil]{babel}
\usepackage[brazil]{varioref}
\usepackage[utf8]{inputenc}
%}}}
%{{{ type output font
\usepackage{
    amsfonts,
    amsmath,
    amsopn,
    amssymb,
    amsthm,
    latexsym
}
\usepackage{indentfirst}
%}}}
%{{{ Question
\usepackage{booktabs}
\usepackage{exsheets}
\usepackage{tasks}
\SetupExSheets[question]{type=exercise}
\SetupExSheets[points]{name=pt/s,number-format=\color{blue}\textbf}
\SetupExSheets{
    counter-format=qu,
    counter-within=section,
    headings=runin,
    solution/print=true
}
%}}}
\usepackage{lipsum}
%}}}
%{{{ Add bib
\addbibresource{bib/book.bib}
%}}}
%{{{ Add New Command
\newcommand\wb[1]{\discretionary{#1}{#1}{#1}}
%}}}
%{{{ Documento
\begin{document}

%{{{ maketitle
\begin{center}
\Large
Universidade Federal Rural de Pernambuco - Recife{\\}
Programa de Pós Graduação em Informática Aplicada{\\}
Departamento de Informática{\\}
\end{center}
\vspace{1 em}

\begin{flushleft}
\large
Disciplina: Processamento de Imagens Digitais{\\}
Professor: Dr. Filipe Cordeiro{\\}
Estudante: Iury Adones{\\}
Data: \today
\end{flushleft}
\vspace{1 em}

\begin{center}
{\Large{Lista de Exercícios}}
\end{center}

\begin{flushleft}
Livro adotado \autocite{GONZALEZ2010}.
\end{flushleft}
%}}}

\section*{Capítulo 4 - Filtragem no Domínio de Frequência}

%{{{ Points
\vspace{1 em}
\begin{center}
\begin{tabular}{l*{\numberofquestions}{c}c}\toprule
    Questão & \ForEachQuestion{\QuestionNumber{#1}\iflastquestion{}{&} } & Total \\ \midrule
    Pontos   & \ForEachQuestion{\GetQuestionProperty{points}{#1}\iflastquestion{}{&} } & \pointssum* \\
    Alcançado  & \ForEachQuestion{\iflastquestion{}{&} } & \\ \bottomrule
\end{tabular}
\end{center}
\vspace{1 em}
%}}}

%{{{ Questions

\settasks{
    counter-format=(tsk[r]),
    label-width=4.5ex,
}

\begin{question}{1}

\noindent
Fazer um resumo com mínimo de 4 páginas do capítulo 4.

\begin{tasks}(5)
    \task Seção 4.1
    \task Seção 4.5.5
    \task Seção 4.7
    \task Seção 4.8
    \task Seção 4.9.0
    \task Seção 4.9.1
    \task Seção 4.9.2
    \task Seção 4.9.3
    \task Seção 4.9.4
    \task Seção 4.9.5
\end{tasks}

\end{question}

\begin{solution}
    \begin{tasks}(1)
        \task Seção 4.1 

            O matemático francês Jean Baptiste Joseph Fourier, contribuiu com
            ideas e formulações matemáticas que foi publicada em 1822 no seu
            livro 1822, \textit{La théorie analitique de la chaleur} (A teoria
            analítica do calor). A contribuição de Fourier foi a afirmação de
            que qualquer função periódica pode ser expressa como a soma de senos
            e/ou cossenos de diferentes frequências, cada uma multiplicada por
            um coeficiente de amplitude diferente, logo está soma passou a ser
            conhecida como a \textit{série de Fourier}. Não importando a
            complexidade da função, sabendo se for periódica e restrita a
            algumas condições matemáticas, então poderá ser representada pela
            soma da série de Fourier. Até funções não periódicas, que em cuja
            área sob uma curva é finita, podem ser expressa como uma integral de
            senos e/ou cossenos multiplicada por uma função de ponderação. A
            formulação é a \textit{Transformada de Fourier}, e sua
            aplicabilidade é ainda maior que a série de Fourier. Nessas
            formulações têm em comum a importante caracteristica de uma função,
            expressa em uma série ou em uma transformada de Fourier, poder ser
            totalmente reconstruída, ou seja, recuperada por meio de um processo
            inverso, sem perda de informação. Portanto essa é uma das
            características mais importantes das representações, pois permite
            trabalhar no ``domínio da Fourier'' e, depois, retornar ao domínio
            original da função sem perder quaisquer informações. As ideais de
            Fourier foram aplicadas na área de difusão de calor, pois permitiram
            a formulação de equações diferenciais que representavam o fluxo de
            calor. No período inicial da década de 1960, foi pesquisado e
            elaborado um algoritmo de transformada rápida de Fourier (FFT, de
            \textit{fast Fourier transform}) que revolucionaram a área de
            processamento de sinais, como também, contribuiu nas modernas
            comunicações eletrônicas, digitalizadores médicos, no entanto
            podemos utilizar tais metodologias no processamento de imagens
            digitais. Assim podemos usar tais técnicas como filtragem no domínio
            da frequência, para realce de imagens digitais.

        \task Seção 4.5.5
        
        A transformação discreta de Fourier 2-D e sua inversa, logo a equação se
        resulta em
        $F(u,v)=\sum_{x=0}^{M-1}\sum_{y=0}^{N-1}f(x,y)e^{-j2{\pi}(ux/M+vy/N)}$,
        sendo $f(x,y)$ uma imagem digital de tamanho $M \times N$. A equação
        deve ser avaliada em termos dos valores das variáveis discretas $u$ e
        $v$ nos intervalos $u = 0, 1, 2, ..., M-1$ e $v = 0, 1, 2, ..., N-1$.
        Dada a transformada $F(u,v)$, podemos obter $f(x,y)$ utilizando a
        \textit{transformada discreta de Fourier inversa} (IDFT), conforme a
        equação
        $f(x,y)=\frac{1}{MN}\sum_{x=0}^{M-1}\sum_{y=0}^{N-1}F(u,v)e^{j2{\pi}(ux/M+vy/N)}$
        para $x = 0, 1, 2, ..., M-1$ e $y = 0, 1, 2, ..., N-1$. Tais equações
        constituem o par de \textit{transformadas discretas de Fourier} 2-D.

        \task Seção 4.7

        página 184  

        \task Seção 4.8

        página 193 

        \task Seção 4.9.0

        página 201

        \task Seção 4.9.1

        página 202

        \task Seção 4.9.2

        página 204

        \task Seção 4.9.3

        página 204

        \task Seção 4.9.4

        página 205

        \task Seção 4.9.5

        página 206

    \end{tasks}
\end{solution}

%}}}

\pagebreak
%{{{ Referências Bibliográficas
\medskip
\printbibliography[
    heading=bibintoc,
    title={Referências Bibliográficas}
]
%}}}
\end{document}
%}}}
