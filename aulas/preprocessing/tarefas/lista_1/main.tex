%{{{ Preamble
\documentclass[12pt,a4paper]{article}
%}}}
%{{{ Packages
%{{{ geometry
\usepackage[
    bottom=3cm,
    left=2cm,
    right=2cm,
    top=3cm,
]{geometry}
%}}}
%{{{ bibliography
\usepackage[
    backend=biber,
    defernumbers=true,
    encoding=latin1,
    style=abnt,
]{biblatex}
\usepackage[autostyle]{csquotes}
%}}}
%{{{ type input font
\usepackage[T1]{fontenc}
\usepackage[brazil]{babel}
\usepackage[brazil]{varioref}
\usepackage[utf8]{inputenc}
%}}}
%{{{ type output font
\usepackage{
    amsfonts,
    amsmath,
    amsopn,
    amssymb,
    amsthm,
    latexsym
}
\usepackage{indentfirst}
%}}}
%{{{ Question
\usepackage{booktabs}
\usepackage{exsheets}
\usepackage{tasks}
\SetupExSheets[question]{type=exercise}
\SetupExSheets[points]{name=pt/s,number-format=\color{blue}\textbf}
\SetupExSheets{
    counter-format=qu,
    counter-within=section,
    headings=runin,
    solution/print=true
}
%}}}
%}}}
%{{{ Add bib
\addbibresource{bib/book.bib}
%}}}
%{{{ Add New Command
\newcommand\wb[1]{\discretionary{#1}{#1}{#1}}
%}}}
%{{{ Documento
\begin{document}

%{{{ maketitle
\begin{center}
\Large
Universidade Federal Rural de Pernambuco - Recife{\\}
Programa de Pós Graduação em Informática Aplicada{\\}
Departamento de Informática{\\}
\end{center}
\vspace{1 em}

\begin{flushleft}
\large
Disciplina: Processamento de Imagens Digitais{\\}
Professor: Dr. Filipe Cordeiro{\\}
Estudante: Iury Adones{\\}
Data: \today
\end{flushleft}
\vspace{1 em}

\begin{center}
{\Large{Lista de Exercícios}}
\end{center}

\begin{flushleft}
Livro adotado \autocite{GONZALEZ2010}.
\end{flushleft}
%}}}

\section*{Capítulo 1 - Introdução}

%{{{ Points
\vspace{1 em}
\begin{center}
\begin{tabular}{l*{\numberofquestions}{c}c}\toprule
    Questão & \ForEachQuestion{\QuestionNumber{#1}\iflastquestion{}{&} } & Total \\ \midrule
    Pontos   & \ForEachQuestion{\GetQuestionProperty{points}{#1}\iflastquestion{}{&} } & \pointssum* \\
    Alcançado  & \ForEachQuestion{\iflastquestion{}{&} } & \\ \bottomrule
\end{tabular}
\end{center}
\vspace{1 em}
%}}}

%{{{ Questions

\begin{question}{0.33}
    Defina Processamento de Imagens Digitais e faça uma associação com as
    disciplinas de Visão Computacional e de Análise de Imagens.
    \begin{tasks}(1)
        \task Por que surgiu a necessidade de realizar o processamento digital
        de imagens?
        \task Quais as vantagens de realizar o processamento digital de imagens?
        \task Cite campos da ciência que utilizam fortemente PDI. Mostre
        exemplos.
    \end{tasks}
\end{question}

\begin{solution}
    Processamento de imagens digitais são procedimentos que utilizam métodos
    matemáticos para diversas representações, tais como uma representação
    matricial e quantizado devido as imagens da realidade serem de caráter
    continuo, a quantização é transformação realizada pelo conversor de sinal
    analógico para digital, ou seja, nos retorna uma matriz com cada elemento da
    matriz têm valores quantizados e que são conhecidos como pixel, logo cada
    operação de processamento de imagem digital numa imagem digital nos
    retornará outra imagem com seus elementos diferentes ou não dependendo da
    aplicação, mas retorna sempre uma imagem, diferente da visão computacional e
    análise de imagens.

    Visão computacional é um sistema ``autônomo'' que por meio das imagens
    digitais é possível emular a visão humana, usando métodos de reconhecimento
    de padrões para identificação de objetos, podendo ser em tempo real ou até
    mesmo próximo a visão humana, mas na visão computacional podemos explorar
    imagens que no espectro humana não seria possível, tais como imagens de
    aquisição de dados do ultra som, porém a visão computacional quando usa uma
    imagem extrai características sem um retorno de uma imagem, logo têm como
    base seus descritores para ajudar na classificação dos objetos. 

    Análise de imagens é um passo que ajuda na tomada de decisões tanto na
    transição de processamento de imagens digitais e na visão computacional,
    pois em sua virtude temos métodos que nos dar informações das imagens, tais
    como para ajudar na tomada de decisão de um filtro, ou na limiarização da
    imagem e entre outros métodos de processamento, até mesmo quando se chega na
    segmentação para ajudar na aprendizagem de maquina para classificação de
    objetos.

    \begin{tasks}(1)
        \task Surgiu para ajudar na visualização das imagens e recuperação das
        imagens com ruídos.
        \task Vantagem na melhora da passagem da informação visual.
        \task Resp
    \end{tasks}
\end{solution}


\settasks{
    counter-format=(tsk[a]),
    label-width=4ex
}
\begin{question}{0.33}
    Descreva os tópicos abordados em cada capítulo do livro de referência:
    \begin{tasks}(2)
        \task aquisição de imagens;
        \task realce de imagens;
        \task restauração de imagens;
        \task processamento de imagens coloridas;
        \task wavelets;
        \task compressão de imagens;
        \task morfologia matemática;
        \task segmentação;
        \task representação e descrição;
        \task reconhecimento de objetos.
    \end{tasks}
\end{question}

\begin{solution}
    \begin{tasks}(2)
        \task aquisição de imagens;
        \task realce de imagens;
        \task restauração de imagens;
        \task processamento de imagens coloridas;
        \task wavelets;
        \task compressão de imagens;
        \task morfologia matemática;
        \task segmentação;
        \task representação e descrição;
        \task reconhecimento de objetos.
    \end{tasks}
\end{solution}



\begin{question}{0.34}
Descreva os passos básicos de um sistema PDI, explicando a utilidade de cada um
deles. Cite ao menos 2 soluções para problemas do cotidiano que podem ser
solucionados utilizando técnicas de PDI. Descreva detalhadamente cada problema,
qual o papel das técnicas de PDI nas resoluções e o que seria feito em cada
etapa básica desses sistemas.
\end{question}

%}}}

\pagebreak
%{{{ Referências Bibliográficas
\medskip
\printbibliography[
    heading=bibintoc,
    title={Referências Bibliográficas}
]
%}}}
\end{document}
%}}}
