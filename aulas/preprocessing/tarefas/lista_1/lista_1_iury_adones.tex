%{{{ Preamble
\documentclass[12pt,a4paper]{article}
%}}}
%{{{ Packages
%{{{ geometry
\usepackage[
    bottom=3cm,
    left=2cm,
    right=2cm,
    top=3cm,
]{geometry}
%}}}
%{{{ bibliography
\usepackage[
    backend=biber,
    defernumbers=true,
    encoding=latin1,
    style=abnt,
]{biblatex}
\usepackage[autostyle]{csquotes}
%}}}
%{{{ type input font
\usepackage[T1]{fontenc}
\usepackage[brazil]{babel}
\usepackage[brazil]{varioref}
\usepackage[utf8]{inputenc}
%}}}
%{{{ type output font
\usepackage{
    amsfonts,
    amsmath,
    amsopn,
    amssymb,
    amsthm,
    latexsym
}
\usepackage{indentfirst}
%}}}
%{{{ Question
\usepackage{booktabs}
\usepackage{exsheets}
\usepackage{tasks}
\SetupExSheets[question]{type=exercise}
\SetupExSheets[points]{name=pt/s,number-format=\color{blue}\textbf}
\SetupExSheets{
    counter-format=qu,
    counter-within=section,
    headings=runin,
    solution/print=true
}
%}}}
%}}}
%{{{ Add bib
\addbibresource{bib/book.bib}
%}}}
%{{{ Add New Command
\newcommand\wb[1]{\discretionary{#1}{#1}{#1}}
%}}}
%{{{ Documento
\begin{document}

%{{{ maketitle
\begin{center}
\Large
Universidade Federal Rural de Pernambuco - Recife{\\}
Programa de Pós Graduação em Informática Aplicada{\\}
Departamento de Informática{\\}
\end{center}
\vspace{1 em}

\begin{flushleft}
\large
Disciplina: Processamento de Imagens Digitais{\\}
Professor: Dr. Filipe Cordeiro{\\}
Estudante: Iury Adones{\\}
Data: \today
\end{flushleft}
\vspace{1 em}

\begin{center}
{\Large{Lista de Exercícios}}
\end{center}

\begin{flushleft}
Livro adotado \autocite{GONZALEZ2010}.
\end{flushleft}
%}}}

\section*{Capítulo 1 - Introdução}

%{{{ Points
\vspace{1 em}
\begin{center}
\begin{tabular}{l*{\numberofquestions}{c}c}\toprule
    Questão & \ForEachQuestion{\QuestionNumber{#1}\iflastquestion{}{&} } & Total \\ \midrule
    Pontos   & \ForEachQuestion{\GetQuestionProperty{points}{#1}\iflastquestion{}{&} } & \pointssum* \\
    Alcançado  & \ForEachQuestion{\iflastquestion{}{&} } & \\ \bottomrule
\end{tabular}
\end{center}
\vspace{1 em}
%}}}

%{{{ Questions

\begin{question}{0.33}
    Defina Processamento de Imagens Digitais e faça uma associação com as
    disciplinas de Visão Computacional e de Análise de Imagens.
    \begin{tasks}(1)
        \task Por que surgiu a necessidade de realizar o processamento digital
        de imagens?
        \task Quais as vantagens de realizar o processamento digital de imagens?
        \task Cite campos da ciência que utilizam fortemente PDI. Mostre
        exemplos.
    \end{tasks}
\end{question}

\begin{solution}
    Processamento de imagens digitais são procedimentos que utilizam métodos
    matemáticos para diversas representações, tais como uma representação
    matricial e quantizado devido as imagens da realidade serem de caráter
    continuo, a quantização é transformação realizada pelo conversor de sinal
    analógico para digital, ou seja, nos retorna uma matriz com cada elemento da
    matriz têm valores quantizados e que são conhecidos como pixel, logo cada
    operação de processamento de imagem digital numa imagem digital nos
    retornará outra imagem com seus elementos diferentes ou não dependendo da
    aplicação, mas retorna sempre uma imagem, diferente da visão computacional e
    análise de imagens.

    Visão computacional é um sistema ``autônomo'' que por meio das imagens
    digitais é possível emular a visão humana, usando métodos de reconhecimento
    de padrões para identificação de objetos, podendo ser em tempo real ou até
    mesmo próximo a visão humana, mas na visão computacional podemos explorar
    imagens que no espectro humana não seria possível, tais como imagens de
    aquisição de dados do ultra som, porém a visão computacional quando usa uma
    imagem extrai características sem um retorno de uma imagem, logo têm como
    base seus descritores para ajudar na classificação dos objetos. 

    Análise de imagens é um passo que ajuda na tomada de decisões tanto na
    transição de processamento de imagens digitais e quanto na visão
    computacional, pois em sua plenitude temos métodos que nos trás informações
    das imagens digitais, ou seja, na área de reconhecimento ou na
    identificação de objetos individuais, de tal modo que nos ajudar na tomada
    de decisão de um filtro e ou na limiarização da imagem, ou até mesmo outro
    método de processamento de imagem digital, podendo se chegar na segmentação
    para recorte de região que contribui para métodos de aprendizagem de máquina
    para classificação de objetos. 
    \begin{tasks}(1)
        \task Surgiu porque com a necessidade de enviar informações visuais, ou
        seja, a imagem, mas com o intuito na visualização das imagens em lugares
        distintos.
        \task Vantagem na melhora da passagem da informação visual, na
        facilitação em diagnósticos, no avanço tecnológicos e científico, e
        contribui na exploração espacial, pois em diversas frequência do
        espectro de onda podemos explorar objetos ou regiões que não são da
        visão humana. Principal vantagem é a informação visual sendo passada de
        forma intuitiva, mesmo que primeira imagem formada tenha ruídos, o
        processamento de imagem ajuda a melhorar a qualidade da imagem e assim
        passar a informação.
        \task Na medicina, utilizam imagem formadas por raios x e podendo
        contribuir no diagnóstico de uma lesão. Na astronomia, utilizado nos
        satélites e em observatórios, onde a coleta pode ser no espectro de luz
        visível ou até mesmo infra vermelho, raios X e gama, contribuindo no
        conhecidos dos corpos celestiais e na exploração espacial.
    \end{tasks}
\end{solution}


\settasks{
    counter-format=(tsk[a]),
    label-width=4ex
}
\begin{question}{0.33}
    Descreva os tópicos abordados em cada capítulo do livro de referência:
    \begin{tasks}(2)
        \task aquisição de imagens;
        \task realce de imagens;
        \task restauração de imagens;
        \task processamento de imagens coloridas;
        \task wavelets;
        \task compressão de imagens;
        \task morfologia matemática;
        \task segmentação;
        \task representação e descrição;
        \task reconhecimento de objetos.
    \end{tasks}
\end{question}

\begin{solution}
    \begin{tasks}(1)
        \task A aquisição da imagem pode ser por uma imagem formada no espectro
        de luz visível ou até mesmo o que o ser humana não observa, tais como
        ultra som, infra vermelho, ultra violeta, raio X, raio Gama e micro
        ondas, para formação da imagem necessário um sensor (conversor de sinal
        analógico para sinal digital) que gera a informação direta ou inderita
        da imagem.
        \task O realce de imagens é o processo de destacar ou intensificar tais
        traços em uma imagem, para tal usa se métodos matemática para realçar um
        objeto em uma imagem ou mesmo destacar um objeto e ofuscar outro.
        \task A restauração de imagens é o processo de melhorar a identificação
        visual das regiões ou objetos, passando por um processo de filtragem de
        ruídos e ou minimização de deficit visuais.
        \task O processamento de imagens coloridas é procedimento de ajustes de
        cores para melhorar ou realçar objetos e ou regiões, e vem ganhando
        espaço ao decorrer e difusão da internet.
        \task O wavelets é usado para representar diversos níveis de resoluções
        de imagem, que ajudar na armazenamento das imagem, devido a compressão,
        mas regiões podem sofre subdivisões em regiões menores.
        \task compressão de imagens que ajuda na transmissão da informação via
        internet, pois comprimindo uma imagem ajuda no tráfego da informação e
        na leitura, mas pode existir perda de informação ou não dependendo do
        métodos utilizado.
        \task A morfologia matemática vem contribui na representação e na
        descrição das formas de uma imagem.
        \task A segmentação ajuda na separação dos objetos, que contribui em uma
        analise mais profunda do objeto individual.
        \task A representação e descrição são usados depois da separação dos
        objetos ou regiões de interesses, sendo assim utilizados os métodos de
        representação e descrição para transformar uma imagem digital com
        diversos pixels em uma informação que possa ser avaliada e testada.
        \task O reconhecimento de objetos é o processo de classificação ou
        identificação de rótulos, tais como já pré definido ou explorado, em uma
        base de conhecimento.
    \end{tasks}
\end{solution}



\begin{question}{0.34}
Descreva os passos básicos de um sistema PDI, explicando a utilidade de cada um
deles. Cite ao menos 2 soluções para problemas do cotidiano que podem ser
solucionados utilizando técnicas de PDI. Descreva detalhadamente cada problema,
qual o papel das técnicas de PDI nas resoluções e o que seria feito em cada
etapa básica desses sistemas.
\end{question}
\begin{solution}
Primeiro o problema, depois um hardware especialista para conversão do problema
em outro problema que é os dados gerados onde um computador consiga armazenar e
que é aproximado ao problema real, ter um sistema de monitoramento para exibição
da informação ou imagem, um software para processamento de imagens que gera ou
melhorar as imagem facilitando a compreensão humana.
Processamento de imagens de carros em alta velocidade, para melhorar a
visualização, assim com objetivo de identificação das placas dos infratores, 
Processamento de escrita para limpar ruídos e ajudar na leitura de pessoas com
dificuldades de compreensão.
No processo de aquisição das imagens dos carros infratores com uma camera 
e sendo armazenadas em uma base de dados, usar métodos de restauração para
movimentos e iluminação, para ajustar a imagem até que fique facilitado a
leitura das placas, depois segmentação das regiões de interesses como as placas.
No processo de aquisição das imagens serão escaneadas, as imagens obtidas de
escritas de alunos e serão armazenadas no computador para formar uma base de
dados, com a idea de utilizar métodos de filtragem, limiarização e restauração
para ruídos, depois utilizar métodos morfológicos e segmentação das regiões de
interesses ou seja as palavras.
\end{solution}

%}}}

\pagebreak
%{{{ Referências Bibliográficas
\medskip
\printbibliography[
    heading=bibintoc,
    title={Referências Bibliográficas}
]
%}}}
\end{document}
%}}}
