%{{{ Preamble
\documentclass[12pt,a4paper]{article}
%}}}
%{{{ Packages
%{{{ geometry
\usepackage[
    bottom=3cm,
    left=2cm,
    right=2cm,
    top=3cm,
]{geometry}
%}}}
%{{{ bibliography
\usepackage[
    backend=biber,
    defernumbers=true,
    encoding=latin1,
    style=abnt,
]{biblatex}
\usepackage[autostyle]{csquotes}
%}}}
%{{{ type input font
\usepackage[T1]{fontenc}
\usepackage[brazil]{babel}
\usepackage[brazil]{varioref}
\usepackage[utf8]{inputenc}
%}}}
%{{{ type output font
\usepackage{
    amsfonts,
    amsmath,
    amsopn,
    amssymb,
    amsthm,
    latexsym
}
\usepackage{indentfirst}
%}}}
%{{{ Question
\usepackage{booktabs}
\usepackage{exsheets}
\usepackage{tasks}
\SetupExSheets[question]{type=exercise}
\SetupExSheets[points]{name=pt/s,number-format=\color{blue}\textbf}
\SetupExSheets{
    counter-format=qu,
    counter-within=section,
    headings=runin,
    solution/print=true
}
%}}}
%}}}
%{{{ Add bib
\addbibresource{bib/book.bib}
%}}}
%{{{ Add New Command
\newcommand\wb[1]{\discretionary{#1}{#1}{#1}}
%}}}
%{{{ Documento
\begin{document}

%{{{ maketitle
\begin{center}
\Large
Universidade Federal Rural de Pernambuco - Recife{\\}
Programa de Pós Graduação em Informática Aplicada{\\}
Departamento de Informática{\\}
\end{center}
\vspace{1 em}

\begin{flushleft}
\large
Disciplina: Processamento de Imagens Digitais{\\}
Professor: Dr. Filipe Cordeiro{\\}
Estudante: Iury Adones{\\}
Data: \today
\end{flushleft}
\vspace{1 em}

\begin{center}
{\Large{Lista de Exercícios}}
\end{center}

\begin{flushleft}
Livro adotado \autocite{GONZALEZ2010}.
\end{flushleft}
%}}}

\section*{Capítulo 2 - Fundamentos}

%{{{ Points
\vspace{1 em}
\begin{center}
\begin{tabular}{l*{\numberofquestions}{c}c}\toprule
    Questão & \ForEachQuestion{\QuestionNumber{#1}\iflastquestion{}{&} } & Total \\ \midrule
    Pontos   & \ForEachQuestion{\GetQuestionProperty{points}{#1}\iflastquestion{}{&} } & \pointssum* \\
    Alcançado  & \ForEachQuestion{\iflastquestion{}{&} } & \\ \bottomrule
\end{tabular}
\end{center}
\vspace{1 em}
%}}}

%{{{ Questions



\begin{question}{1}
    \begin{tasks}(1)
        \task Discuta sobre o espectro eletromagnético da luz visível.
        \task Se podemos enxergar apenas uma pequena parte do espectro visível, como conseguimos ver imagens em infravermelho, por exemplo?
        \task Desenhe a estrutura do olho humano, descreva seus principais componentes e como a imagem é formada.
        \task Desenhe a estrutura (básica) de uma câmera CCD. Descreva seus principais componentes e como a imagem é obtida.
        \task Qual a analogia que pode ser realizada entre os componentes da câmera e do olho?
    \end{tasks}
\end{question}

\begin{solution}
    \begin{tasks}(1)
        \task Resp
        \task Resp 
        \task Resp
    \end{tasks}
\end{solution}


\settasks{
    counter-format=(tsk[a]),
    label-width=4ex
}
\begin{question}{1}
Suponha uma imagem onde cada pixel é representado por um byte, mas todos os seus pixels (320x240) encontram-se em uma faixa de 16 tons de cinza (por exemplo, entre 100 e 115). Proponha outra forma de representar esta imagem, qual o possível ganho no armazenamento desta imagem? Extrapole a sua ideia para um sistema de representação em que você não conhece a priori a quantidade de tons de cinza das imagens de entrada e avalie os ganhos máximos e mínimos possíveis na codificação da imagem em comparação com um bitmap (um byte por pixel).
\end{question}

\begin{solution}
    \task resposta;
\end{solution}

\begin{question}{1}
A imagem resultante da operação booleana AND entre duas imagens de entrada deverá conter média menor ou igual a menor média das imagens de entrada. Você concorda com essa afirmativa? Explique.
\end{question}

\begin{solution}
    \task resposta;
\end{solution}

\begin{question}{1}
A imagem resultante da operação booleana OR entre duas imagens de entrada deverá conter média menor ou igual a menor média das imagens de entrada. Você concorda com essa afirmativa? Explique.
\end{question}

\begin{solution}
    \task resposta;
\end{solution}

\begin{question}{1}
Suponha que uma área plana com centro em $({x_0},{y_0})$ seja iluminada por
uma fonte de luz com distribuição de intensidade 
$i(x,y) = Ke^{-[(x-{x_0})^{2}+(y-{y_0})^{2}]}$

Suponha, para fins de simplificação, que a refletância da área seja constante e
igual a 1,0 e que K=255. Se a imagem resultante for digitalizada com k bits de
resolução de intensidade e o olho puder detectar uma mudança subida de oito
níveis de intensidade entre pixels adjacentes, qual valor de k causará um falso
contorno visível?
\end{question}

\begin{solution}
    \task resposta;
\end{solution}

\begin{question}{1}
A mediana, $\alpha$, de um conjunto de números é tal que metade dos valores do
conjunto está baixo de $\alpha$ e a outra metade acima dele. Por exemplo, a
mediana de um conjunto de valores \{2,3,8,20,21,25,31\} é 20. Demonstre que um
operador que computa a mediana de uma subimagem de área, S, é não-linear.
\end{question}

\begin{solution}
    \task resposta;
\end{solution}

% P. Exercícios de programação

% Para os exercícios seguintes, você poderá implementar em qualquer linguagem de programação, mas não deverá utilizar nenhuma biblioteca de processamento de imagens para realizar as operações.

% P.1 A rotação de uma imagem pode ser realizada multiplicando a posição de cada pixel pela matriz de rotação R. Dada uma imagem binária I, uma matriz identidade 100x100, realize as seguintes operações:
%     a) Gere a imagem A a partir da rotação I em 45 graus;
%     b) Gere a imagem B a partir da rotação de A em 45 graus;
%     c) Gere a imagem C a partir da rotação de I em 90 graus;

% $R(\theta) = [[cos(\theta) -sin(\theta)][sin(\theta) cos(\theta)]]$

% Compare B e C. Se B e C não forem idênticas encontre a causa do problema que fez com que isso ocorresse, proponha ainda uma solução para esse problema e analise as limitações da solução proposta.


% P.2 Escreva um programa capaz de reduzir o número de níveis de intensidade de uma imagem de 256 para 2, em inteiros da base 2. O número de níveis de intensidade necessários deve ser uma variável de entrada do seu programa. Obs: Use a figura “Fig0221.tif”


% P.3 a) Escreva um programa capaz de dar zoom out (redução) e zoom in (ampliação) ou reduzir a imagem por replicação de pixels. Assuma que o fator de zoom é inteiro. Utilize a figura “Fig0220.tif” para reduzir a imagem por um fator de 10.
% b) Use o programa para dar zoom na imagem em (B) para a resolução original. Explique as razões das diferenças.

%}}}

\pagebreak
%{{{ Referências Bibliográficas
\medskip
\printbibliography[
    heading=bibintoc,
    title={Referências Bibliográficas}
]
%}}}
\end{document}
%}}}
